% !TEX root = ../AC52010-Report.tex
%
\chapter[Concepts]{Concepts: This text is here to test a very long title, to simulate the line break behaviour, to show that an extremely long title also works}
\label{sec:concepts}

%\chapter[Short chapter name]{Long chapter name ...}

\cleanchapterquote{Users do not care about what is inside the box, as long as the box does what they need done.}{Jef Raskin}{about Human Computer Interfaces}

\Blindtext[2][1]

\section{Concepts Section 1}
\label{sec:concepts:sec1}

\Blindtext[2][2]

\section{Concepts Section 2}
\label{sec:concepts:sec2}

\blindmathtrue
\blindtext
 
\blindmathfalse
\blindtext


\section{Concepts Section 3}
\label{sec:concepts:sec3}

%\Blindtext[4][2]
 
\blindlist{env}[x]

\blindlist{itemize}[12]
%2
%3
	
\blinditemize
\blindenumerate
\blinddescription



\section{Conclusion}
\label{sec:concepts:conclusion}


\begin{table}[]    \centering
\begin{tabular}{|c|r|c|c|}
\hline
Q & ref  & file  & result \\
\hline\hline
1 & PCA dimensionality reduction     & PCA.m & PCA.pdf \\
1 & clustering                       & PCA.m & PCA.pdf \\
1 & scatter plot                     & PCA.m & PCA.pdf \\
1 & Discussion                       & \S \ref{sec:concepts:conclusion} &   \\
2 & LDA dimensionality reduction     & LDA.m & LDA.pdf \\
2 & scatter plot    & LDA.m & LDA.pdf \\
2 & Discussion    & \S \ref{sec:concepts:conclusion} &  \\
3 & SVM with a linear kernel & SVM.m & SVM.pdf \\
3 & SVM with a RBF kernel & SVM.m & SVM.pdf \\
  & (with one hidden layer) & & \\
  & (NN system generated code) & myNeuralNetworkFunction.m & \\
  &  & createfigure\_netROC.m & \\
  &  & createfigure\_netconfmat.m & \\
3 & parameter tuning & demo\_libsvm\_ & demo\_libsvm\_ \\
  &                  & crossvalidation.m & crossvalidation.pdf \\
\hline
\label{randomstuff}   \caption{Random Stuff}
\end{tabular}
\end{table}


\Blindtext[2][1]
