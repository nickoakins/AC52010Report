

% AC52010


% The general arrangement of the report should be as follows:

%    Title Page (Project Title, Author, Name of Degree & University, Supervisor, Date)
%    Executive Summary of Project (1 Page)
%    Declaration: signed by you (1 page)

%    "I declare that the special study described in this dissertation has been carried out and the dissertation composed by me, and that the dissertation has not been accepted in fulfilment of the requirements of any other degree or professional qualification."
%    Certificate, signed by your supervisor (1 page)

%    "I certify that (your full name) has satisfied the conditions of the Ordinance and Regulations and is qualified to submit this dissertation in application for the degree of Master of Science."
%    Acknowledgements
%    Table of Contents
%    List of Figures
%    Main body (typically 5 or 6 chapters)
%    References
%    Appendices

%The report should be written in a formal style. All pages should be numbered, the first page of chapter 1 being page 1. All references should be cited in the main body of the report. The report should demonstrate that the student has used appropriate tools to support the development process and that verification and validation have been applied at all stages.

%The main body of the report should address the following:

%    Introduction: An explanation of the problem and the objectives of the project.
%    Background: A (usually brief) review of relevant literature and products to establish the context of the project.
%    Requirements specification: A specification of the problem and an explanation of how the student arrived at this specification. An initial work schedule including an overall project plan with time-scales, deliverables and resources.
%    Design: This should include the design method, design process and outcome. Design decisions and trade-offs should be described e.g. when selecting algorithms, data structures and implementation environments or when designing for usability.
%    Implementation and Testing: A description of production, testing and debugging. A demonstration that the specification has been satisfied.
%    Evaluation: You should carry out formal user evaluations and report on them.
%    A description of the functionality and interfaces of the completed system.
%    Appraisal: A critical appraisal of the project indicating the rationale for design/implementation decisions, lessons learnt during the course of the project and an evaluation (with hindsight) of the final product and the process of its production (including a review of the plan and any deviations from it). The project should be placed in a wider context and this could include the scientific, technical, commercial, social and ethical context.
%    Summary and Conclusions.
%    Recommendations for future work.

% Appendices

% The main body of the report should read as a self-contained document. However, appendices can be used for necessary supporting documentation, such as software, source code, a user manual, minutes of your meetings, forms used for evaluations etc. Appendices should only be submitted electronically. 



% !TEX root = ../thesis-example.tex
% !TEX root = ../AC52010-Report.tex

