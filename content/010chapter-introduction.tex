% !TEX root = ../AC52010-Report.tex
%
\chapter{Introduction}
\label{sec:intro}

%\cleanchapterquote{You can’t do better design with a computer, but you can speed up your work enormously.}{Wim Crouwel}{(Graphic designer and typographer)}

\cleanchapterquote{All of old. Nothing else ever. Ever tried. Ever failed. No matter. Try again. Fail again. Fail better.}{Samuel Beckett}{Worstward Ho (1983)}

%    Introduction: An explanation of the problem and the objectives of the project.
%    Background: A (usually brief) review of relevant literature and products to establish the context of the project.
%    Requirements specification: A specification of the problem and an explanation of how the student arrived at this specification. An initial work schedule including an overall project plan with time-scales, deliverables and resources.
%    Design: This should include the design method, design process and outcome. Design decisions and trade-offs should be described e.g. when selecting algorithms, data structures and implementation environments or when designing for usability.
%    Implementation and Testing: A description of production, testing and debugging. A demonstration that the specification has been satisfied.
%    Evaluation: You should carry out formal user evaluations and report on them.
%    A description of the functionality and interfaces of the completed system.
%    Appraisal: A critical appraisal of the project indicating the rationale for design/implementation decisions, lessons learnt during the course of the project and an evaluation (with hindsight) of the final product and the process of its production (including a review of the plan and any deviations from it). The project should be placed in a wider context and this could include the scientific, technical, commercial, social and ethical context.
%    Summary and Conclusions.
%    Recommendations for future work.


\Blindtext[2][2]



\section{Motivation and Problem Statement}
\label{sec:intro:motivation}

\Blindtext[3][1] \cite{Jurgens:2000,Jurgens:1995,Miede:2011,Kohm:2011,Apple:keynote:2010,Apple:numbers:2010,Apple:pages:2010}

\section{Results}
\label{sec:intro:results}

\Blindtext[1][2]

\subsection{Some References}
\label{sec:intro:results:refs}
\cite{WEB:GNU:GPL:2010,WEB:Miede:2011}

\section{Thesis Structure}
\label{sec:intro:structure}

\textbf{Chapter \ref{sec:related}} \\[0.2em]
\blindtext

\textbf{Chapter \ref{sec:system}} \\[0.2em]
\blindtext

\textbf{Chapter \ref{sec:concepts}} \\[0.2em]
\blindtext

\textbf{Chapter \ref{sec:concepts}} \\[0.2em]
\blindtext

\textbf{Chapter \ref{sec:conclusion}} \\[0.2em]
\blindtext
