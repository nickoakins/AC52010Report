% !TEX root = ../AC52010-Report.tex
%
\chapter{System}
\label{sec:system}

\cleanchapterquote{Innovation distinguishes between a leader and a follower.}{Steve Jobs}{(CEO Apple Inc.)}

\Blindtext[2][1]



\section{Requirements specification}
\label{sec:system:spec}

%    Requirements specification: A specification of the problem and an explanation of how the student arrived at this specification. An initial work schedule including an overall project plan with time-scales, deliverables and resources.

\Blindtext[1][2]

\begin{figure}[htb]
	\includegraphics[width=\textwidth]{gfx/Clean-Thesis-Figure}
	\caption{Figure example: \textit{(a)} example part one, \textit{(c)} example part two; \textit{(c)} example part three}
	\label{fig:system:example1}
\end{figure}

\Blindtext[1][2]

\section{Design}
\label{sec:system:design}

%    Design: This should include the design method, design process and outcome. Design decisions and trade-offs should be described e.g. when selecting algorithms, data structures and implementation environments or when designing for usability.

\Blindtext[1][2]

\begin{figure}[htb]
	\includegraphics[width=\textwidth]{gfx/Clean-Thesis-Figure}
	\caption{Another Figure example: \textit{(a)} example part one, \textit{(c)} example part two; \textit{(c)} example part three}
	\label{fig:system:example2}
\end{figure}

\Blindtext[2][2]

\section{Implementation and Testing}
\label{sec:system:imptest}

%    Implementation and Testing: A description of production, testing and debugging. A demonstration that the specification has been satisfied.

\section{Evaluation}
\label{sec:system:evaluation}

%    Evaluation: You should carry out formal user evaluations and report on them.

%    A description of the functionality and interfaces of the completed system.


\Blindtext[4][2]

\section{Conclusion}
\label{sec:system:conclusion}

\Blindtext[2][1]
